\documentclass[paper=a4, fontsize=11pt]{scrartcl}
\usepackage{enumerate}
\usepackage{amsmath}
\usepackage{amssymb}
\usepackage{tikz}
\newcommand{\parens}[1]{ \left( #1 \right) }
\begin{document}
\noindent Willy Xiao and Kevin Eskici \\ STAT 221 \\Pset 5\\ Nov 18, 2014
\begin{enumerate}[\text{question }1.]
  \item
    \begin{enumerate}[1]
      \item INSERT KEVIN'S PLOT
      \item INSERT KEVIN'S PLOT
      \item
        \begin{enumerate}[a.]
          \item
          \begin{align*}
            Q(\theta, \theta^k) &= E(l(\theta|X)|Y, \theta^k) \\
            &= E(-T/2\log{|\Sigma|} - 1/2\sum{(x_t-\lambda)'\Sigma^{-1}(x_t - \lambda)}) \\
            &= -T/2\log{|\Sigma|} - 1/2*\sum{E(m_t^k - \lambda)'\Sigma^{-1}(m_t^k - \lambda)) + tr(\Sigma^{-1}R^{k})} \\
            &= -T/2(\log{|\Sigma|} + tr(\Sigma^{-1}R^k)) - 1/2\sum_{t=1}^{T}{(m\_t^{(k)} - \lambda)'\Sigma^{-1}(m_t^{(k)} - \lambda)}
          \end{align*}
          \item Pseudocode:
            \begin{enumerate}[I.]
              \item Set $x = \vec{0}$, $v = \vec{0}$
              \item For each $y_t$ do \\
                \indent $v := Concat(v, y_{t,1})$ \\
                \indent $x := x + y_t$
              \item Set $\lambda_0 = \overrightarrow{1'(x/w)}$, $\phi_0 = var(v)/mean(v)$.
              \item Initialize $\theta^{(0)} = \theta_0 = (\lambda_0, \phi_0)$.
              \item Until $\theta^{(k+1)} = \theta^{(k)}$ do \\
                \begin{tabbing}
                  \hspace{1cm} $Q := (\theta, \theta^{(k)})$ do: \\
                  \hspace{2cm} $\Sigma := \phi*diag(\lambda_1^c, \ldots, \lambda_I^c)$. \\
                  \hspace{2cm} $\Sigma^{(k)} := \phi^{(k)}*diag((\lambda_1^{(k)})^c, \ldots, (\lambda_I^{(k)})^c)$ \\
                  \hspace{2cm} $R^{(k)} := \Sigma^{(k)} - \Sigma^{(k)}A'solve(A\Sigma^{(k)}A')A\Sigma^{(k)}$ \\
                  \hspace{2cm} $S := $ sum for each $t$: \\
                  \hspace{3cm} $m := \lambda^{(k)} + \Sigma^{(k)}A'solve(A\Sigma^{(k)}A')(y_t - A\lambda^{(k)})$ \\
                  \hspace{3cm} return $(m - \lambda)'\Sigma(m - \lambda)$. \\
                  \hspace{2cm} return $-\frac{T}{2}\log{|\Sigma|} + tr(solve(\Sigma)R^{(k)}) - (1/2)S$. \\
                  \hspace{1cm} $\theta^{k} := \theta^{(k+1)}$ \\
                  \hspace{1cm} $\theta^{(k+1)} := optim(Q)$
                \end{tabbing}
            \end{enumerate}
        \end{enumerate}
      \item For the local iid model, instead of initializing with all $y_t$ we initialize with $y_t \in [y_{t - w/2}, y_{t+w/2}]$ and we also sum only over those $t \in [t - w/2, t + w/2]$. INSERT PLOT FOR LOCALLY\_IID\_EM.
      \item The refined model is, after choosing V just \\
      \begin{align*}
        p(\eta_t|\widetilde{Y_t}) &= p(\eta_t|\widetilde{Y}_{t-1}, Y_t) \propto p(\eta_t|\widetilde{Y}_{t-1})p(Y_t|\eta_t) \\
        p(\eta_t|\widetilde{Y}_{t-1}) &\sim N(\hat{\eta}_{t-1}, \hat{\Sigma}_{t-1})
      \end{align*}
      Where
      \begin{align*}
        \hat{\Sigma}_{t-1} &= g''(\eta_t)^{-1} = -\hat{\Sigma}^{-1}_{t|t-1} + \partial^2{\log{p}}/\partial{\eta_t}^2 \\
        & \text{ and the second derivative is defined on page 1067. } \\
        & \text { We just used the hessian function. }
      \end{align*}
      \item INSERT PLOT OF REFINED
      \item In general the Tebaldi-West method should be pretty similar to the CaoYu method in producing the final outputs. In many ways, Tebaldi-West will be even better because it doesn't approximate the poisson with a normal model that CaoYu does. The issue; however, is that the Tebladi-West algorithm will take a long time to run (even for the case of the Markov-Gibbs algorithm). In a sense you can think that Tebaldi-West throws a ton of darts at a board to find the maximium likelihood for $x_t$; while CaoYu is starting at the edge of the boarding and walking closer and closer to the best $x_t$ (ie the EM is moving towards the better values).
      \item INSERT PLOTS
    \end{enumerate}
\end{enumerate}
\end{document}
